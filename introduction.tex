
\section{Introduction}
\label{sec:introduction}

Over the last several decades we have observed the development of a 
large collection of specification and modeling languages and 
associated methodologies, and tools. Their purpose is to 
support modeling of requirements and high-level designs before 
programming is initiated. Agile approaches advocate to avoid 
explicit modeling entirely and suggest to go directly to coding. 
Other approaches advocate avoiding manual “coding” in a programming 
language entirely and suggest instead the generation of code 
directly from the models. This way modeling languages replace 
programming languages.  We can divide these modeling languages into 
formal specification languages (formal methods), usually focusing 
on textual languages based on mathematical logic and set theory, 
and associated proof tools (theorem provers, model checkers, etc.), 
and on the other hand model-based engineering languages (UML, 
SysML, Modelica, Mathematica, …), focusing more on design, code 
generation and simulation. Many of these modeling languages have 
similarities with programming languages.

In parallel, and frankly seemingly independent, we have seen the 
development of numerous new programming languages. Few languages 
have had the success of C, which still today is the main 
programming language for embedded systems. The success is so 
outstanding that nearly no progress wrt. praxis has been made in 
this domain (embedded programming) since the 1970ties.  In 
application programming a collection of new languages came ago such 
as Ada, C++, Eiffel etc. At the same time we have seen several 
high-level languages appear targeting the softer side of software 
engineering (such as web-programming, user interfaces, scripting), 
including languages such as Java, JavaScript, Ruby, Python and 
Scala.  More academic languages include Haskell and the ML family, 
including OCaml.

In this paper we attempt to argue for a unified view of modeling
and programming. This view can in the extreme be considered a call
for a single universal formalism for modeling and programming any
form of system. However, we fully understand that this is not likely
to happen. What we intend is to fuel an {\em effort} to at least consider
merging efforts to the extent feasible. We believe that the model-based
engineering community can learn from the formal methods and 
programming communities, and vice versa. Note that even if a single
formalism would appear, there will always be alternatives, just like
there are multiple programming languages (evolution continues). 

This paper is organized as follows.
In Section \ref{sec:trends} we give a brief overview of some 
of the trends in modeling and programming, that we consider important. 
In Section \ref{sec:modeling-as-programming} we illustrate with examples how modeling can be perceived as programming. 
%In Section \ref{sec:concerns} we suggest discussion points to be reflected on
%when considering a unified approach.
Finally Section \ref{sec:conclusion} outlines
brief discussion points to be reflected on
when considering a unified approach, as well
as a conclusion.

