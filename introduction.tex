
\section{Introduction}
\label{sec:introduction}

Over the last several decades we have observed the development of a 
large collection of specification and modeling languages and 
associated methodologies, and tools. Their purpose is to 
support formulation of requirements and high-level designs before 
programming is initiated. Agile approaches advocate to avoid 
explicit modeling entirely and suggest to go directly to coding. 
Other approaches advocate avoiding manual ``coding'' in a 
programming 
language entirely and suggest instead the generation of code 
directly from the models. This way modeling languages replace 
programming languages.  We can divide modeling languages into 
formal specification languages (formal methods), usually focusing 
on textual languages based on mathematical logic and set theory, 
and associated proof tools (theorem provers, model checkers, etc.), 
and on the other hand model-based engineering languages (UML, 
SysML, Modelica, Mathematica, ...), focusing more on design, code 
generation and simulation. Many of these modeling languages have 
similarities with programming languages.

In parallel, and frankly seemingly independent, we have seen the 
development of numerous new programming languages. Few languages 
have had the success of C, which still today is the main 
programming language for embedded systems. The success is so 
outstanding that nearly no progress wrt. praxis has been made in 
this domain (embedded programming) since the 1970ties, although 
some richer languages appeared soon after C in this domain, such as 
C++, Ada, and Eiffel. These later languages for example
all have module systems, which C does not. We have seen several high-level languages appear that target the softer side of software 
engineering (such as web-programming, user interfaces, scripting), 
including languages such as Java, JavaScript, Ruby, Python and 
Scala.  More academic languages include Haskell and the ML family, 
including OCaml.

There is seemingly a strict difference between a modeling
language and a programing language. For a
programming language we always assume a notion of {\em 
executability} and {\em computability}. Programming languages 
are restricted to concepts that can be
executed. Put differently, programming languages put emphasis on 
the {\em ``how''}, the algorithms for solving problems.
A specification and modeling language in principle should rather 
focus on the {\em ``what''}. A mathematical way of phrasing this is 
that specifications should ideally be {\em predicates} on solutions 
(executions for example). Intuitively, one may also argue that 
there are modeling tasks which do not directly aim at programing, 
for instance if we model a business process independent of the 
question which parts should be carried out by machines. This is 
modeling, which seems far away from programming. It might be 
interesting to bring it into a form which is closer to programming 
if we want to simulate or automatically analyze such models. But 
here there seems to be a boundary. Programming means computability. 
Modeling can be more general. Finally, at a more technical level, 
in programming, at least when we work in general purpose 
programming languages, we have to deal with non termination and the 
concept of undefined \cite{broy-chaos-2006}. 
In a number of modeling approaches such 
concepts like undefined are avoided. Here again 
there is an interesting challenge in a unifying
view onto modeling and programming. We would have to manage to
introduce the concepts of undefined into modeling, representing
nonterminating expressions in programming.  Some attempts have been
made in	this direction though, for example 3-valued logic as 
found in VDM.

In spite of these perceived differences, the similarities between
modeling languages and programming languages are obvious, which
suggests a unifying view. For example, many logics support the 
notions of local variables with bounded scopes and syntactic 
expressions similar to programming languages. Many modeling 
languages even offer programming constructs, such as mutable 
variables, 
assignment statements, and looping (while) statements, and of course recursion. Furthermore, some of the modeling 
languages, such as UML, are deeply influenced by programming 
languages wrt. how models are structured. In particular, the idea 
of object-oriented modeling is taken from the concept of object-
oriented programming. It is even considered one of the strong sides 
of object-orientation, that one can have a unified view of 
object-oriented specification, object-oriented design, and 
object-oriented programming. In summary, the concepts that 
are used in modeling and the concepts that are used in 
programming are so closely related that it is beneficial to attempt 
a unified view.

In this paper we attempt to argue for such a unified view 
of modeling and programming. This view can in the extreme be 
considered a call
for a single universal formalism for modeling and programming any
form of system. This is done by high-lighting some trends in 
modeling and programming, and by programming some example
models in the Scala programming language, a high-level formalism
suited for this purpose.
However, we fully understand that such a unification faces many
obstacles, some of which are non-technical. 
What we intend is to fuel an {\em effort} to at least consider
merging efforts to the extent feasible. We believe that the model-based
engineering community can learn from the formal methods and 
programming communities, and vice versa. Note that even if a single
formalism would appear, there will always be alternatives, just like
there are multiple programming languages (evolution continues). 

The paper is organized as follows.
In Section \ref{sec:trends} we give a brief overview of some 
of the trends in modeling and programming, that we consider important. 
In Section \ref{sec:modeling-as-programming} we illustrate with examples how modeling can be perceived as programming. 
%In Section \ref{sec:concerns} we suggest discussion points to be reflected on
%when considering a unified approach.
Finally Section \ref{sec:conclusion} outlines
brief discussion points to be reflected on
when considering a unified approach, as well
as a conclusion.
