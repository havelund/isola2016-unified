
\section{Discussion and Conclusion}
\label{sec:conclusion}


\label{sec:concerns}

A unified modeling and programming framework has to satisfy 
quite different and contradicting goals.
First of all, it has to represent the concepts of the 
application domain at an adequate level of 
abstraction such that the specialities of the applications 
are directly represented and not covered by awkward implementation 
concepts.  
Second it has to address the structuring of algorithms and data 
structures in a way such that programs stay understandable, 
modular and support the most important methods of structured 
program development. And finally it has to allow addressing 
specific implementation properties of execution machines including 
their operating systems, such that it can be controlled how the 
implementation uses resources and exploits the possibilities of 
the execution platform and its hardware. An obvious problem here 
of course is to what extent the particular application domain 
influences the programming language, and to what extent this 
is true for the execution platform and efficiency concerns as
well. In the following we shall briefly mention some of the
elements to consider when imagining a unified approach to
modeling and programming.


\subsubsection{Target domains}

We can observe three major domains of interest, namely modeling; 
programming of non-embedded systems, such as web applications, 
including scripting; and finally programming of embedded/cyber-
physical systems. It is clear that these three domains till date 
have been addressed by different communities and different 
languages. The question is to what extent these
quite different domains could be targeted with the same formalism.
Note that the different modeling and programming languages used on
different targets have an overwhelming number of language constructs in
common, to an extent where this question at least needs to be answered
in a scientific manner rather than in an opinionated emotional manner.

\subsubsection{Predicate specifications}

A formalism must generally support specifying properties as 
predicates rather than only as algorithms. Predicate-oriented techniques 
include design-by-contract, including pre- and post-conditions, as 
well as state invariants. Such can for example be found in Eiffel 
as  well as in SPARK. This concept can be carried 
further to for example include behavioral sequence specification, 
such as temporal logics, sequence diagrams, etc.
Note, however, that many models are very operational in
nature, and hence can very well best be formulated as state 
machines, or programs (data structures and algorithms). 

\subsubsection{Programming in the large}

A formalism must support programming in the large, and in general
provide good modularization and component-based development. One cannot discuss components without discussing concurrency. Concurrency is an essential part of modern programming, especially 
considering the emergence of multi-core computers. However, 
concurrency is important at the modeling level as well, where it 
can serve as a natural way to describe interacting agents. 
Important concepts include agent systems, message passing based 
communication, parallel data structures (programming concurrent 
without knowing it), and distributed programming.

\subsubsection{High-level programming}

A formalism must support high-level programming as found in modern
programming languages. The elegance  of functional programming has 
been praised many times. Nevertheless its breakthrough is only 
recent. In contrast, object-oriented programming, 
which in particular addresses encapsulation and reuse, has been very 
popular for decades now. A language such as Scala 
integrates the two paradigms nicely, as even early versions 
of LISP (CLOS) did. 
Functional programming means for example functions as 
values (lambda abstractions) and pattern matching, and of course 
reliance on recursion. Functional programming is by some considered 
the best approach to use multi-core systems due to no shared state 
updates. A key feature of VDM was the introduction of elegant syntax for collections, such as sets, lists and maps. These days such concepts are introduced in languages mostly as libraries. Fortress has built-in syntax for these very similar to VDM.
There should be easy ways of iterating through collections – to avoid indexing problems, as well as support for parallel computation over such.
A formalism should be statically typed, although with type inference, and with allowance for going type less
in clearly defined regions to support scripting. 
Decades of experience in strong type systems should be harvested, including more recent  topics such as dependent types, session types, and units.

\subsubsection{Low-level programming}

A formalism must support low-level programming.
Embedded programing often means: no dynamic memory 
allocation after initialization, no garbage collection, some 
knowledge of memory layout, even to the point where computation 
with addresses is used to improve speed. This again means use of 
low level programming languages such as C. C, however, allows for 
memory errors and makes programmers less effective as they would 
otherwise be were they allowed to program in higher-level 
languages. We need to satisfy the needs encountered by typical C 
programmers, including offering comparable speed and memory 
control. This includes support for hardware control and
targeting specific execution platforms 


\subsubsection{Continuous mathematics}

A formalism must support modeling of cyber-physical systems. That
is: physical systems controlled by computer programs. To model (not program) a cyber-physical system, there is a need for describing
continuous behavior using continuous mathematics, including
for example differential equations, as supported by for example
Modelica \cite{?}. Modelica is an object-oriented  
language for modeling systems containing mechanical, electrical, electronic, hydraulic, thermal, control, electric power or process-oriented subcomponents. Models can be simulated.
Similar tools are: Mathematica \cite{?} and
MATLAB \cite{?}.


\subsubsection{Domain-specific languages}

A formalism must support definition of DSLs (Domain-Specific 
Languages). A key to modeling and programming is to capture 
the relevant concepts of the application domain. UML for example
supports meta-modeling and profiles, used for defining (graphical)
DSLs. The programming language community is still somewhat behind
in this respect. It should be easy for developers to define new
DSLs either external stand-alone, or internal extending the
modeling/programming formalism.

\subsubsection{Visualization}

A formalism must be visualizable.
Visualization is of extreme importance, as demonstrated by the 
relative success of formalisms such as UML and SysML in the 
engineering community, when compared 
to formal methods textual notations. People generally are more at 
ease looking at a 2-dimensional illustration when they encounter a 
problem specification for the first
time, and it eases communication. Whether visualization is useful 
as well for entering
and maintaining specifications is another matter. We do believe 
that visualization and 
text should match up, such that modification of one leads to 
immediate change in the 
other. Visualization is normally static, rendering static models as
pictures. However, visualization can also be dynamic, of program 
executions, illustrating how behavior evolves.

\subsubsection{Analysis}

A formalism must be analyzable.
A key concept in a combined modeling and programming environment is 
the support for advanced analysis of models/programs, including, 
but also beyond, what is normally supported in standard programming 
environments. This ranges from basic built-in support for unit 
testing, over advanced testing capabilities, including test input 
generation and monitoring, to concepts such as static analysis, 
model checking, theorem proving and symbolic execution. A core 
requirement, however, must be the practicality of these solutions. 
The main emphasis should be put on automation. The average user 
should be able to benefit from automated verification, without 
having to do manual proofs. However, support for manual theorem 
proving should also be possible, for example for core critical 
algorithms. Integration of static and dynamic analysis will be 
desirable: verify what is practically feasible, and test (monitor) 
the remaining proof obligations.

\subsection{...}

\subsubsection{Conclusion}

We have in this paper outlined some views on the potential in 
combining modeling and programming, supported by analysis 
capabilities such as static analysis, model checking, theorem 
proving, monitoring, and testing. We believe that the time is right 
for the formal methods/modeling and programming language 
communities to join forces. To some extent this is already 
happening in the small. However, we believe that we are standing in 
front of a major wave of research creating a united foundation for 
modeling, programming and verification. A cynical argument is that 
this is all obvious, which may very well be true. 